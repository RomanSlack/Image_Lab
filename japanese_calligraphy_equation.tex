\documentclass[12pt]{article}
\usepackage{amsmath}
\usepackage{amssymb}
\usepackage{geometry}
\geometry{a4paper, margin=1in}
\usepackage{xcolor}

\title{Japanese Calligraphy Effect: Complete Mathematical Model}
\author{Image Effects Lab}
\date{}

\begin{document}

\maketitle

\section{Complete Brush Stroke Function}

The Japanese calligraphy effect can be expressed as a composite function that maps an input image $I(x,y)$ to a rendered brush stroke at position $(x,y)$ with various parameters. Here is the complete mathematical model:

\subsection{The Master Equation}

For each stroke seed position $(c_x, c_y)$ and parameter $t \in [0,1]$ along the stroke:

\begin{equation}
\boxed{
\mathbf{S}(c_x, c_y, t) = \begin{bmatrix} x(t) \\ y(t) \\ W(t) \\ \mathbf{C}(c_x, c_y) \\ \alpha(t) \end{bmatrix}
}
\end{equation}

where each component is defined as follows:

\subsection{Position Along Stroke Path}

\begin{align}
x(t) &= c_x + \cos(\Theta) \cdot L \cdot (t - 0.5) + \cos\left(\Theta + \frac{\pi}{2}\right) \cdot \mathcal{C}(t) \\
y(t) &= c_y + \sin(\Theta) \cdot L \cdot (t - 0.5) + \sin\left(\Theta + \frac{\pi}{2}\right) \cdot \mathcal{C}(t)
\end{align}

with curve offset:
\begin{equation}
\mathcal{C}(t) = \sin(t\pi) \cdot c_{\text{random}} \cdot \frac{L}{4}, \quad c_{\text{random}} \sim \mathcal{U}(-0.15, 0.15)
\end{equation}

\subsection{Stroke Angle $\Theta$}

The stroke angle depends on the stroke type classification:

\begin{equation}
\Theta(c_x, c_y) = \begin{cases}
\theta_{\text{user}}(c_x, c_y) + \mathcal{U}(-0.15, 0.15) & \text{if directional region} \\[0.5em]
\arctan2(G_y, G_x) & \text{if bark region and } ||\nabla I|| > 10 \\[0.5em]
\frac{\pi}{2} + \mathcal{U}(-0.3, 0.3) & \text{if bark region and } ||\nabla I|| \leq 10 \\[0.5em]
\arctan2(G_y, G_x) + \frac{\pi}{2} + \mathcal{U}(-0.2, 0.2) & \text{if edge pixel} \\[0.5em]
\mathcal{U}(0, 2\pi) & \text{otherwise (random)}
\end{cases}
\end{equation}

where the gradient magnitude is:
\begin{equation}
||\nabla I|| = \sqrt{G_x^2 + G_y^2}, \quad G_x = \frac{\partial I}{\partial x}, \quad G_y = \frac{\partial I}{\partial y}
\end{equation}

\subsection{Stroke Length $L$}

\begin{equation}
L(c_x, c_y) = \begin{cases}
L_{\text{base}} \cdot \mathcal{U}(1.5, 2.2) \cdot m_w & \text{if directional region} \\
L_{\text{base}} \cdot \mathcal{U}(2.0, 2.8) \cdot m_w & \text{if bark region} \\
L_{\text{base}} \cdot 1.3 \cdot m_w & \text{if edge pixel} \\
L_{\text{base}} \cdot m_w & \text{otherwise}
\end{cases}
\end{equation}

where $m_w$ is the width multiplier for each stroke type.

\subsection{Brush Size (Variable Mode)}

When variable brush size is enabled:

\begin{align}
d(c_x, c_y) &= \frac{E(c_x, c_y)}{255} \\
b_{\text{base}} &= \frac{b_{\min} + b_{\max}}{2} \\
r_{\text{size}} &= \frac{b_{\max} - b_{\min}}{2} \\
b(c_x, c_y) &= b_{\text{base}} + (1 - d) \cdot r_{\text{size}} \cdot v_{\text{size}} \cdot \mathcal{U}(0.9, 1.1) \\
B(c_x, c_y) &= \text{clip}(b(c_x, c_y), b_{\min}, b_{\max})
\end{align}

where $E(c_x, c_y)$ is the Canny edge strength, and $v_{\text{size}}$ is the size variation parameter.

\subsection{Width Taper Function}

The width along the stroke varies with parameter $t$:

\begin{equation}
w(t) = \begin{cases}
\frac{t}{0.2} \cdot 0.9 & \text{if } t < 0.2 \quad \text{\color{blue}(gradual start)} \\[0.5em]
0.9 + 0.1 \sin\left(\frac{(t-0.2)\pi}{0.4}\right) & \text{if } 0.2 \leq t \leq 0.6 \quad \text{\color{blue}(middle section)} \\[0.5em]
1.0 - \frac{t-0.6}{0.4} \cdot \tau & \text{if } t > 0.6 \quad \text{\color{blue}(tapered tail)}
\end{cases}
\end{equation}

Final width at position $t$:
\begin{equation}
W(t) = \max\left(1, \lfloor B(c_x, c_y) \cdot w(t) \rfloor \right)
\end{equation}

\subsection{Color Sampling}

Color is sampled from the original image region:
\begin{equation}
\mathbf{C}(c_x, c_y) = \frac{1}{|R|} \sum_{(i,j) \in R} I(i, j) + \mathcal{U}(-5, 6)^3
\end{equation}

where $R$ is the local region around $(c_x, c_y)$, and $\mathcal{U}(-5, 6)^3$ represents a 3D uniform random vector for color variation.

\subsection{Ink Bleed Alpha Transparency}

\begin{equation}
\alpha(t) = \begin{cases}
220 + \mathcal{U}(-10, 10) & \text{if } t \leq 0.7 \\[0.5em]
\left(220 + \mathcal{U}(-10, 10)\right) \cdot \left(1 - \frac{t-0.7}{0.3} \cdot 0.4\right) & \text{if } t > 0.7 \text{ and ink bleed enabled}
\end{cases}
\end{equation}

\section{Complete Unified Form}

Combining all components, the complete brush stroke rendering function is:

\begin{equation}
\boxed{
\begin{aligned}
\mathbf{S}(c_x, c_y, t) = &\Bigg[
c_x + \cos(\Theta(c_x, c_y)) \cdot L(c_x, c_y) \cdot (t - 0.5) \\
&\quad + \cos\left(\Theta(c_x, c_y) + \frac{\pi}{2}\right) \cdot \sin(t\pi) \cdot \mathcal{U}(-0.15, 0.15) \cdot \frac{L(c_x, c_y)}{4}, \\[0.8em]
&c_y + \sin(\Theta(c_x, c_y)) \cdot L(c_x, c_y) \cdot (t - 0.5) \\
&\quad + \sin\left(\Theta(c_x, c_y) + \frac{\pi}{2}\right) \cdot \sin(t\pi) \cdot \mathcal{U}(-0.15, 0.15) \cdot \frac{L(c_x, c_y)}{4}, \\[0.8em]
&\max\left(1, \lfloor B(c_x, c_y) \cdot w(t) \rfloor \right), \\[0.8em]
&\frac{1}{|R|} \sum_{(i,j) \in R} I(i, j) + \mathcal{U}(-5, 6)^3, \\[0.8em]
&\alpha(t) \Bigg]
\end{aligned}
}
\end{equation}

where all sub-functions $\Theta$, $L$, $B$, $w$, and $\alpha$ are as defined above.

\section{Edge Detection (Preprocessing)}

Before stroke generation, edge detection is performed:

\begin{align}
E(x, y) &= \text{Canny}(I(x,y), \theta_{\text{low}}=50, \theta_{\text{high}}=150) \\
G_x(x,y) &= \text{Sobel}_x(I(x,y)) = I * K_x \\
G_y(x,y) &= \text{Sobel}_y(I(x,y)) = I * K_y \\
\theta_{\text{edge}}(x,y) &= \arctan2(G_y(x,y), G_x(x,y))
\end{align}

where $K_x$ and $K_y$ are the Sobel kernels:
\begin{equation}
K_x = \begin{bmatrix} -1 & 0 & 1 \\ -2 & 0 & 2 \\ -1 & 0 & 1 \end{bmatrix}, \quad
K_y = \begin{bmatrix} -1 & -2 & -1 \\ 0 & 0 & 0 \\ 1 & 2 & 1 \end{bmatrix}
\end{equation}

\section{Parameters}

\begin{itemize}
\item $b_{\min}, b_{\max}$: Minimum and maximum brush sizes
\item $L_{\text{base}}$: Base stroke length
\item $\tau$: Tail taper amount $\in [0,1]$
\item $v_{\text{size}}$: Size variation sensitivity $\in [0,1]$
\item $\mathcal{U}(a,b)$: Uniform random distribution on interval $[a,b]$
\item $I(x,y)$: Input image at pixel $(x,y)$
\end{itemize}

\end{document}
